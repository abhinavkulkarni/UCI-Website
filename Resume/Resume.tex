\documentclass[a4paper,20pt]{article}

\usepackage[total={6.5in,8.75in}, top=1.2in, left=0.9in]{geometry}
\usepackage{fancyhdr}
\usepackage{hyperref}
\usepackage{parcolumns}

\pagestyle{fancy}
% Remove page numbers
\fancyfoot{}

\setlength{\headheight}{28pt}
\chead{%
  \fontsize{20.74pt}{24pt}\selectfont{}%
  Abhinav Mukund Kulkarni\\%
  \fontsize{8pt}{9.6pt}\selectfont{}%
  3900 30C, Parkview Lane, Irvine, CA, 92612, United States; +1 512 658 7456; %
  \href{mailto:abhinavkulkarni@gmail.com}{\nolinkurl{abhinavkulkarni@gmail.com}}; %
  \href{http://ics.uci.edu/~amkulkar}{WWW}%
}
\renewcommand{\headrulewidth}{0.4pt}

\begin{document}
\begin{parcolumns}[nofirstindent, colwidths={1=.15\linewidth}]{2}
  \colchunk[1]{%
  EDUCATION\\*
  \\*
  \\*
  \vspace*{-6pt}
  \\*
  }
  \colchunk[2]{%
  \textbf{University of California, Irvine}\hfill 2011 - 2013 (Expected)\\*
  Master of Science, Computer Science, 4.0/4.0\\*
  \vspace*{-6pt}
  \\*
  \textbf{National Institute of Technology, Tiruchirapalli, India}\hfill 2005 - 2009\\*
  Bachelor of Technology, Computer Science, 8.34/10.0
  }
\end{parcolumns}
\vspace{11pt}

\begin{parcolumns}[nofirstindent, colwidths={1=.15\linewidth}]{2}
  \colchunk[1]{%
  COURSEWORK\\*
  }
  \colchunk[2]{%
  Fall '11: Intermediate Statistics A, Machine Learning, Data Structures\\*
  Winter '12: Intermediate Statistics B, Statistical Methodology II, Probabilistic Learning, Information Retreival\\*
  Spring '12: Intermediate Statistics C, Statistical Methodology III, Graphical Models
  }
\end{parcolumns}
\vspace{\baselineskip}

\begin{parcolumns}[nofirstindent, colwidths={1=.15\linewidth}]{2}
  \colchunk[1]{%
  WORK\\*
  EXPERIENCE\\*
  }
  \colchunk[2]{%
  \textbf{Microsoft Corporation}\emph{, Search Technology Center}\hfill Hyderabad, India\\*
  \emph{Software Developer Engineer}\hfill October 2009 - August 2011\\*
  The focus was on improving relevance of algorithmic search overall as well as in specific segments by employing statistical methods to mine information from user click patterns from Internet Explorer and Microsoft Toolbar user click logs. My work on relevance of local search queries (containing locations) and offline computation of geo-specific popularity of websites resulted in new features incorporated in Bing search engine.\\*
  \\*
  \textbf{Indian Institute of Technology, Madras}\emph{, Computer Vision Lab}\hfill Chennai, India\\*
  \emph{Project Associate}\hfill August 2009 - October 2009
  The project involved development of highly scalable multi-camera surveillance system that allows face recognition and object tracking from live video feed. The task was to accommodate multiple cameras and allow real-time object search in databases.
  }
\end{parcolumns}
\vspace{\baselineskip}
\begin{parcolumns}[nofirstindent, colwidths={1=.15\linewidth}]{2}
  \colchunk[1]{%
  \noindent PROJECT\\*
  WORK
  }
  \colchunk[2]{%
  \textbf{Heritage Provider Network Health Prize}\hfill Fall 2011, UC Irvine\\*
  Participated in \href{http://www.heritagehealthprize.com/c/hhp}{HHP Prize competition} hosted by popular data modeling and prediction platform \href{http://www.kaggle.com}{Kaggle}. Implemented machine learning algorithms to identify patients who will be admitted to a hospital within next year using historical claims, drug and lab data.\\*
  \\*
  \textbf{Chemistry Corpus Analysis}\hfill Winter 2012, UC Irvine\\*
  We worked on the hypothesis that Chemistry is more closed in terms of collaboration among researchers than other Physical Sciences such as Physics and Mathematics. We analysed the graph of authors of publications from reputed journals induced by co-authorship and affiliation relations and contrasted Chemistry with other Physical Sciences.\\*
  \\*
  \textbf{Bachelor's Thesis: }Simulated Study of Generalized Distributed Scheduler\\*
  Advisor:\href{http://www.nitt.edu/home/academics/departments/cse/faculty/leela/}{Prof. Leela Velusamy}\hfill December 2008 - April 2009, NIT Tichy\\*
  Simulated study of Generalized Distributed Task Scheduler (GDS) in Grid Environment which allocate resources to the tasks classified in the real time categories such as critical and non-critical by migrating tasks for better schedulability.
  }
\end{parcolumns}
\vspace{\baselineskip}

\begin{parcolumns}[nofirstindent, colwidths={1=.15\linewidth}]{2}
  \colchunk[1]{%
  SKILLS
  }
  \colchunk[2]{%
  \textbf{Languages: }C, C++, R; \textbf{Packages/Tools: }Eclipse, Visual Studio, MATLAB, Octave
  }
\end{parcolumns}
\vspace{\baselineskip}

\begin{parcolumns}[nofirstindent, colwidths={1=.15\linewidth}]{2}
  \colchunk[1]{%
  AWARDS
  }
  \colchunk[2]{%
  Team mentioned in the special meritorious list of ACM ICPC 2008, South Asia Region.
  }
\end{parcolumns}
\vspace{\baselineskip}

\begin{parcolumns}[nofirstindent, colwidths={1=.15\linewidth}]{2}
  \colchunk[1]{%
  LEADERSHIP\\*
  }
  \colchunk[2]{%
  Marketing Head for Fine Arts Society of NIT Tiruchirapalli.\\*
  Event Manager for CSurf ’09, the online programming contest conducted in Vortex ’09, the annual departmental symposium of CSE department of NIT Tiruchirapalli.
  }
\end{parcolumns}
\vspace{\baselineskip}

\begin{parcolumns}[nofirstindent, colwidths={1=.15\linewidth}]{2}
  \colchunk[1]{%
  EXTRA\\*
  CURRICULAR
  }
  \colchunk[2]{%
  Winner of gold medal in swimming contests held in sports events of NIT Tiruchirapalli.\\*
  Community service through National Service Scheme (NSS) involving forestation.
  }
\end{parcolumns}
\end{document}
